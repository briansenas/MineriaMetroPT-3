\documentclass[12pt,letterpaper]{article}
\usepackage[framemethod=tikz]{mdframed}
\usepackage[utf8]{inputenc} %Spanish input
\usepackage[T1]{fontenc} % Use 8-bit encoding that has 256 glyphs
\usepackage[spanish, es-tabla]{babel} % Selecciona el español para palabras introducidas automáticamente, p.ej. "septiembre" en la fecha y especifica que se use la palabra Tabla en vez de Cuadro
\usepackage{fullpage}
\usepackage[top=2cm, bottom=4.5cm, left=2.5cm, right=2.5cm]{geometry}
\usepackage{lastpage}
\usepackage{enumerate}
\usepackage[inline]{enumitem}
\usepackage{fancyhdr}
\usepackage{xcolor}
%
\usepackage[sorting=none]{biblatex}
\addbibresource{citas.bib}
%
\usepackage{csquotes}
\usepackage{cellspace}
\setlength{\cellspacetoplimit}{5pt}
\setlength{\cellspacebottomlimit}{5pt}
\usepackage{hhline}
\usepackage{listings}
\usepackage{hyperref}
\usepackage{titletoc,tocloft}
\usepackage{float,subfig}
\setlength{\cftsubsecindent}{2cm}
\setlength{\cftsubsubsecindent}{4cm}
\dottedcontents{section}[1.5em]{}{1.3em}{.6em}
%\usepackage[nodisplayskipstretch]{setspace}
%
\graphicspath{ {./imgs/} } %Drawing the background pic
\usepackage{tikz}
\newcommand{\tikzmark}[1]{\tikz[baseline,remember picture] \coordinate (#1) {};}
\usetikzlibrary{positioning}
\usetikzlibrary{shadows,arrows.meta} % For adding edges label
\usetikzlibrary{calc}
\usepackage{eso-pic}
\AddToShipoutPictureBG{%
    \begin{tikzpicture}[remember picture, overlay]
        \node[opacity=.15, inner sep=0pt]
            at(current page.center){\includegraphics[scale=1.5]{logo-ugr2}};
    \end{tikzpicture}%
}

% \numberwithin{equation}{section} % Number equations within sections (i.e. 1.1, 1.2, 2.1, 2.2 instead of 1, 2, 3, 4)
% \numberwithin{figure}{section} % Number figures within sections (i.e. 1.1, 1.2, 2.1, 2.2 instead of 1, 2, 3, 4)
% \numberwithin{table}{section} % Number tables within sections (i.e. 1.1, 1.2, 2.1, 2.2 instead of 1, 2, 3, 4)

\hypersetup{%
    colorlinks=true,
    linkcolor=[rgb]{0.2, 0.3, 0.5},
    urlcolor=black,
    citecolor=black,
    linkbordercolor={0 0 1}
}

\renewcommand\lstlistingname{Código:}
\renewcommand\lstlistlistingname{Código:}
\def\lstlistingautorefname{Brian SS.}
%
%
\newcommand{\horrule}[1]{\rule{\linewidth}{#1}} % Create horizontal rule command with 1 argument of height
\definecolor{codegreen}{rgb}{0,0.6,0}
\definecolor{codegray}{rgb}{0.5,0.5,0.5}
\definecolor{codepurple}{rgb}{0.58,0,0.82}
\definecolor{backcolour}{rgb}{0.95,0.95,0.92}
%
\lstset{language=python,basicstyle=\linespread{1.1}\ttfamily\footnotesize,
    xleftmargin=0.0cm, frame=t, framesep=0.15cm, framerule=0pt, tabsize=4,
    showspaces=false, showstringspaces=false,showlines=true,
    keywordstyle=\color{blue}\ttfamily,
    stringstyle=\color{red}\ttfamily,
    commentstyle=\color{gray}\ttfamily,
    morecomment=[l][\color{magenta}]{\#}
}
%
\setlength{\parindent}{0.0in}
\setlength{\parskip}{0.05in}
%
%% Edit these as appropriate
\newcommand\course{Ciencia de Datos e Ingenieria de Computadores}
\newcommand\hwnumber{1}                  % <-- homework number
\newcommand\NetIDa{Brian}           % <-- NetID of person #1
\newcommand\NetIDb{Sena Simons}           % <-- NetID of person #1
%
\pagestyle{fancyplain}
\headheight 35pt
\lhead{\NetIDa}
\lhead{\NetIDa\\\NetIDb}                 % <-- Comment this line out for problem sets (make sure you are person #1)
\chead{\hspace{-2cm}\textbf{\large Preprocesamiento y Clasificación}}
\rhead{\course \\ \today}
\lfoot{\scriptsize\LaTeX}
\cfoot{\hyperlink{Indice}{Volver al índice}}
\rfoot{\small\thepage}
\headsep 1.5em
%
\renewcommand*\contentsname{Índice}
%
\author{Brian Sena Simons} % Nombre y apellidos
%
\date{\normalsize\today} % Incluye la fecha actual
%
\begin{document}
%
\begin{titlepage}
\begin{figure}[H]
    \vspace{-1.3cm}
    \begin{center}
        \includegraphics[width=0.75\textwidth]{Etsiit}
    \end{center}
\end{figure}
\vspace{1.3cm}
\centering
\normalfont \normalsize
\textsc{\textbf{Minería de Datos} \\ \vspace{.15cm} Master en Ciencia de Datos e Ingeniería de Computadores \\ \vspace{.15cm} Universidad de Granada} \\ [25pt] % Your university, school and/or department name(s)
    \horrule{0.5pt} \\[0.4cm] % Thin top horizontal rule
    \huge Preprocesamiento y Clasificación\\ % The assignment title
    \horrule{2pt} \\[0.5cm] % Thick bottom horizontal rule

\begin{minipage}{0.4\textwidth}
    \begin{flushleft}\large
        \emph{Autores:} \\
         ----------------------- \\
        \vspace{.15cm}
        Brian Sena Simons. \\
        Miguel Garcia Lopez. \\
        Álvaro Santana Sánchez. \\ 
        Ana Fuentes Rodríguez.

    \end{flushleft}
\end{minipage}
\begin{minipage}{0.4\textwidth}
    \vspace{-2.2cm}
     \begin{flushright}\large
         Grupo: \\
         ----------------------- \\
         Data Mavericks.
    \end{flushright}
\end{minipage}
\end{titlepage}


\hypertarget{Indice}{}
\tableofcontents
\newpage
\section{Introducción.}
Se ha realizado un análisis y comparativa entre diferentes modelos para la detección de anomalías y predicción de vida útil restante (RUL por sus siglas en inglés)
en compresores del sector ferroviario. Para ello, se ha utilizado el conjunto de datos (dataset) ``MetroPT-3''~\cite{MetroPT-3}.
Está publicado en ``UCI Machine Learning Repository''~\cite{UCIMLR} y, según la descripción, MetroPT-3~\cite{MetroPT-3} es un conjunto de datos multivariantes de series temporales. Los datos provienen de sensores analógicos y digitales 
instalados en un compresor de tren, que miden 15 señales como presiones, corriente del motor, temperatura del aceite y señales eléctricas de las válvulas de entrada de aire. 
La información fue registrada a una frecuencia de 1 Hz entre febrero y agosto de 2020 (véase Tabla~\ref{tab:DatosBasicos})

\begin{table}[!ht]
    \centering
    \begin{tabular}{|l|l|l|c|c|c|c|c|c|}
    \hline
        \textbf{Variable} & \textbf{Tipo} & \textbf{Mín.} & \textbf{Q1} & \textbf{Q2} & \textbf{Media} & \textbf{Q3} & \textbf{Máx.} \\ \hline
TP2               & Numérico      & -0.032        & -0.014      & -0.012      & 1.368         & -0.010      & 10.676        \\ \hline
TP3               & Numérico      & 0.730         & 8.492       & 8.960       & 8.985         & 9.492       & 10.302        \\ \hline
H1                & Numérico      & -0.036        & 8.254       & 8.784       & 7.568         & 9.374       & 10.288        \\ \hline
DV\_pressure      & Numérico      & -0.032        & -0.022      & -0.020      & 0.05596       & -0.018      & 9.844         \\ \hline
Reservoirs        & Numérico      & 0.712         & 8.494       & 8.960       & 8.985         & 9.492       & 10.300        \\ \hline
Oil\_temperature  & Numérico      & 15.40         & 57.77       & 62.70       & 62.64         & 67.25       & 89.05         \\ \hline
Motor\_current    & Numérico      & 0.020         & 0.040       & 0.045       & 2.050         & 3.808       & 9.295         \\ \hline
COMP              & Numérico      & 0.000         & 1.000       & 1.000       & 0.837         & 1.000       & 1.000         \\ \hline
DV\_eletric       & Numérico      & 0.000         & 0.000       & 0.000       & 0.1606        & 0.000       & 1.000         \\ \hline
Towers            & Numérico      & 0.000         & 1.000       & 1.000       & 0.9198        & 1.000       & 1.000         \\ \hline
MPG               & Numérico      & 0.000         & 1.000       & 1.000       & 0.8327        & 1.000       & 1.000         \\ \hline
LPS               & Numérico      & 0.000         & 0.000       & 0.000       & 0.00342       & 0.000       & 1.000         \\ \hline
Pressure\_switch  & Numérico      & 0.000         & 1.000       & 1.000       & 0.9914        & 1.000       & 1.000         \\ \hline
Oil\_level        & Numérico      & 0.000         & 1.000       & 1.000       & 0.9042        & 1.000       & 1.000         \\ \hline
Caudal\_impulses  & Numérico      & 0.000         & 1.000       & 1.000       & 0.9371        & 1.000       & 1.000         \\ \hline
    \end{tabular}
    \caption{Información básica de los diferentes tipos de datos presentes en MetroPT-3~\cite{MetroPT-3}}
    \label{tab:DatosBasicos}
\end{table}

Este conjunto de datos tiene como objetivo principal mejorar la detección de fallos y la predicción de mantenimiento. 
Aunque no contiene etiquetas directas, se dispone de informes de fallos que permiten evaluar la efectividad de los algoritmos de detección de anomalías, predicción de fallos y estimación de RUL (véase la Tabla~\ref{tab:Reportes}).

\begin{table}[!ht]
    \centering
\begin{tabular}{|c|l|l|r|l|}
\hline
\textbf{Número} & \textbf{Inicio}       & \textbf{Fin}         & \textbf{Duración} & \textbf{Importancia} \\ \hline
1            & 4/12/2020 11:50          & 4/12/2020 23:30           & 700                & Alta              \\ \hline
2            & 4/18/2020 00:00          & 4/18/2020 23:59           & 1440               & Alta              \\ \hline
3            & 4/19/2020 00:00          & 4/19/2020 01:30           & 90                 & Alta              \\ \hline
4            & 4/29/2020 03:20          & 4/29/2020 04:00           & 40                 & Alta              \\ \hline
5            & 4/29/2020 22:00          & 4/29/2020 22:20           & 20                 & Alta              \\ \hline
6            & 5/13/2020 14:00          & 5/13/2020 23:59           & 599                & Alta              \\ \hline
7            & 5/18/2020 05:00          & 5/18/2020 05:30           & 30                 & Alta              \\ \hline
8            & 5/19/2020 10:10          & 5/19/2020 11:00           & 50                 & Alta              \\ \hline
9            & 5/19/2020 22:10          & 5/19/2020 23:59           & 109                & Alta              \\ \hline
10           & 5/20/2020 00:00          & 5/20/2020 20:00           & 1200               & Alta              \\ \hline
11           & 5/23/2020 09:50          & 5/23/2020 10:10           & 20                 & Alta              \\ \hline
12           & 5/29/2020 23:30          & 5/29/2020 23:59           & 29                 & Alta              \\ \hline
13           & 5/30/2020 00:00          & 5/30/2020 06:00           & 360                & Alta              \\ \hline
14           & 6/01/2020 15:00          & 6/01/2020 15:40           & 40                 & Alta              \\ \hline
15           & 6/03/2020 10:00          & 6/03/2020 11:00           & 60                 & Alta              \\ \hline
16           & 6/05/2020 10:00          & 6/05/2020 23:59           & 839                & Alta              \\ \hline
17           & 6/06/2020 00:00          & 6/06/2020 23:59           & 1439               & Alta              \\ \hline
18           & 6/07/2020 00:00          & 6/07/2020 14:30           & 870                & Alta              \\ \hline
19           & 7/08/2020 17:30          & 7/08/2020 19:00           & 90                 & Alta              \\ \hline
20           & 7/15/2020 14:30          & 7/15/2020 19:00           & 270                & Media            \\ \hline
21           & 7/17/2020 04:30          & 7/17/2020 05:30           & 60                 & Alta              \\ \hline
\end{tabular}
    \caption{
    Intervalos de tiempo con problemas en la compresión del aire.
    Nos permite evaluar la capacidad de detección anomalías de nuestros modelo.}
    \label{tab:Reportes}
\end{table}

Además, se recomienda utilizar el primer mes de datos para entrenar modelos, dejando el resto para las pruebas, permitiendo también la formación incremental si fuera necesario.


\section{Análisis Exploratorio de Datos.}
\subsection{Introducir aquí las visualizaciones y comentarios EDA}
\textbf{Insertar aquí el EDA}
\subsection{Introducir aquí los intervalos ``raros'' hallados}
Durante el análisis EDA y prepración del conjunto de datos de entrenamiento y evaluación se han recogido nuevas anomalías, veáse la Tabla [] y Figura[].
Sería interesante volver a consultar con un experto del campo para que valide dichas anomalías. No obstante, presentan un perfil suficientemente cercano al de las anomalías clasificadas.
Por ello, consideramos oportuno la inclusión de dichos ejemplos como anomalías para ayudar a paliar el bajo número de ejemplo de casos positivos....

Tras observar y analizar el conjunto de datos, seguimos un acercamiento similar a [citar SAEVAE/RuleBased]  para tratar a la serie temporal. 
Para ello....encontramos que el tiempo medio de ambos motores se acercan a los resultados de [RuleBased] (véase Tabla ...), donde....
Hay que tener en cuenta la presencia de valores perdidos (sensores no envían la información) y los saltos temporales en los ciclos que introducen las anomalías.

Teniendo en cuenta que el tiempo medio de ciclo es de 1260 segundos, utilizaremos un intervalo el doble de este como nuestra ventana deslizante.
Logramos mejorar el tiempo mínimo de información para clasificación frente a [ruleBased] y acercanos al obtenido en [SAE/VAE].
[mencionar imagen tal]

\section{Regresión Logística.}
\section{Máquinas de Vectores de Soporte.}
\section{Clasificador Bayesiano.}
\section{Árboles de clasificación.}
\section{Gradient Boosting.}
\section{Stacking.}
\section{AdaBoost.}
\section{Bagging.}
\printbibliography
\end{document}
